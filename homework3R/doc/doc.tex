\documentclass[a4paper,twoside=false,abstract=false,numbers=noenddot,
titlepage=false,headings=small,parskip=half,version=last]{scrartcl}

\usepackage[utf8]{inputenc}
\usepackage[T1]{fontenc}
\usepackage[english]{babel}

\usepackage[colorlinks=true, pdfstartview=FitV,
linkcolor=black, citecolor=black, urlcolor=blue]{hyperref}
\usepackage{verbatim}
\usepackage{graphicx}
\usepackage{multirow}

\usepackage{tikz}
\usetikzlibrary{matrix}

\usepackage{amsmath}
\usepackage{amsthm}
\usepackage{amssymb}
\usepackage{amsfonts}

\theoremstyle{definition}
\newtheorem{exercise}{Exercise}

\theoremstyle{remark}
\newtheorem*{solution}{Solution}
\newtheorem*{remark}{Remark}

\newtheorem{theorem}{Theorem}[section]
\newtheorem{lemma}[theorem]{Lemma}
\newtheorem{proposition}[theorem]{Proposition}
\newtheorem{corollary}[theorem]{Corollary}

\newcommand{\NN}{\ensuremath{\mathbb{N}}}
\newcommand{\ZZ}{\ensuremath{\mathbb{Z}}}
\newcommand{\QQ}{\ensuremath{\mathbb{Q}}}
\newcommand{\RR}{\ensuremath{\mathbb{R}}}
\newcommand{\CC}{\ensuremath{\mathbb{C}}}
\newcommand{\GG}{\ensuremath{\mathcal{G}}}
\newcommand{\Fourier}{\ensuremath{\mathcal{F}}}
\newcommand{\Laplace}{\ensuremath{\mathcal{L}}}

\DeclareMathOperator{\Hom}{Hom}
\DeclareMathOperator{\End}{End}
\DeclareMathOperator{\im}{im}
\DeclareMathOperator{\id}{id}

\renewcommand{\labelenumi}{(\alph{enumi})}

\author{Jim Holmström - 890503-7571}
\title{Groups and Rings - SF2729}
\subtitle{Homework3 (Rings)}

\begin{document}

\maketitle
\thispagestyle{empty}

\begin{exercise}
{\bf
Let $R$ be a commutative ring with unity of prime characteristic $p$. Show that
the map $\phi_p:R\rightarrow R$ given by $\phi_p(a)=a^p$ is a homomorphism.
}
\end{exercise}
\begin{solution}
In a commutative ring 
\begin{equation}
    (a+b)^n=\sum \binom{n}{k} a^ib^{n-i}
\end{equation}
holds. Where
\begin{equation}
    \binom{n}{k} = \frac{n!}{k!(n-k)!}
\end{equation}
and we can see that we have if $n$ is prime.
\begin{equation}
    p|\binom{p}{i} ~~ \forall i \in \left[1,p-1\right]
\end{equation}
and hence the terms
\begin{equation}
    \binom{p}{i}a^ib^{p-i}=0
\end{equation}
in a commutative ring with characteristic $p$. This gives us the "freshman's dream"
\begin{equation}
    (a+b)^p=a^p+b^p
\end{equation}
With these facts its easy to show that $\phi$ is a homomorphism.
\begin{equation}
    \phi_p(a+b)=(a+b)^p=a^p+b^p=\phi_p(a)+\phi_p(b)
\end{equation}
and trivially since $R$ is commutative
\begin{equation}
    \phi_p(ab)=(ab)^p=a^pb^p=\phi_p(a)\phi_p(b)
\end{equation}
and thus $\phi_p$ is a homomorphism. \qed

\end{solution}

%-----------------------
\begin{exercise}
{\bf
Prove that if $F$ is a field, every proper nontrivial prime ideal of
$F\left[x\right]$ is maximal.
}
\end{exercise}
\begin{solution}
By theorem 26.24 from the book every ideal of $F\left[x\right]$ is principal.
Suppose $\langle f(x) \rangle \neq {0}$ is a proper prime ideal of
$F\left[x\right] \Rightarrow$ all polynomial in $\langle f(x) \rangle$ has
degree equal or greater than the degree of $f(x)$. Thus if we have $f=gh$
in $F\left[x\right]$ with both $g,h$ degrees less than $f$, neither $g$ nor $h$
can be in $\langle f(x) \rangle$. Which is a contradiction on the fact that we
said it was an prime ideal $\Rightarrow$ no such factorization in
$F\left[x\right]$ could exist $\Rightarrow$ $f$ irreducible in
$F\left[x\right]$ and by theorem 26.25 from the book $\langle f(x) \rangle$ is
therefore a maximal ideal of $F\left[x\right]$. $\qed$

\end{solution}

%-----------------------
\end{document}
