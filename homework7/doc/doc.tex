\documentclass[a4paper,twoside=false,abstract=false,numbers=noenddot,
titlepage=false,headings=small,parskip=half,version=last]{scrartcl}

\usepackage[utf8]{inputenc}
\usepackage[T1]{fontenc}
\usepackage[english]{babel}

\usepackage[colorlinks=true, pdfstartview=FitV,
linkcolor=black, citecolor=black, urlcolor=blue]{hyperref}
\usepackage{verbatim}
\usepackage{graphicx}
\usepackage{multirow}

\usepackage{tikz}
\usetikzlibrary{matrix}

\usepackage{amsmath}
\usepackage{amsthm}
\usepackage{amssymb}
\usepackage{amsfonts}

\theoremstyle{definition}
\newtheorem{exercise}{Exercise}

\theoremstyle{remark}
\newtheorem*{solution}{Solution}
\newtheorem*{remark}{Remark}

\newtheorem{theorem}{Theorem}[section]
\newtheorem{lemma}[theorem]{Lemma}
\newtheorem{proposition}[theorem]{Proposition}
\newtheorem{corollary}[theorem]{Corollary}

\newcommand{\NN}{\ensuremath{\mathbb{N}}}
\newcommand{\ZZ}{\ensuremath{\mathbb{Z}}}
\newcommand{\QQ}{\ensuremath{\mathbb{Q}}}
\newcommand{\RR}{\ensuremath{\mathbb{R}}}
\newcommand{\CC}{\ensuremath{\mathbb{C}}}
\newcommand{\GG}{\ensuremath{\mathcal{G}}}
\newcommand{\Fourier}{\ensuremath{\mathcal{F}}}
\newcommand{\Laplace}{\ensuremath{\mathcal{L}}}

\DeclareMathOperator{\Hom}{Hom}
\DeclareMathOperator{\End}{End}
\DeclareMathOperator{\im}{im}
\DeclareMathOperator{\id}{id}

\renewcommand{\labelenumi}{(\alph{enumi})}

\author{Jim Holmström - 890503-7571}
\title{Groups and Rings - SF2729}
\subtitle{Homework 7}

\begin{document}

\maketitle
\thispagestyle{empty}

\begin{exercise}
{\bf
How many $p$-Sylow subgroups the group $A_5$ has for $p=3,5,7$
}
\end{exercise}
\begin{solution}

\begin{theorem}[Third Sylow Teorem]
   $p$ prime $\bigwedge \: p \mid |G|$ 
   $\Rightarrow$
   $\#\{$Sylow $p$-subgroups$\} \equiv 1 \:( mod \: p)$
   $\bigwedge$
   $\#\{$Sylow $p$-subgroups$\} \mid |G|$ 
\end{theorem}

$S=\{a : a \mid |A_5|=5!/2\}=\{1,2,3,4,5,6,10,12,15,20,30,60\}$

\section{$p=3$}
$\{ a \in S: a \equiv 1 \:(mod\: 3) \}=\{1,4,10\}$\\
Subgroups generate by cycles of order $3$ in $A_5$ is Sylow $3$-subgroups.\\
$\langle (123) \rangle, \langle (124) \rangle, \langle (125) \rangle,
 \langle (134) \rangle, \langle (135) \rangle, \ldots $ \\
Which is more then $4$ distinct Sylowgroups which results in:\\ 
\underline{
    $\#\{$Sylow $3$-subgroups$\}=10$    
}

\section{$p=5$}
$\{ a \in S: a \equiv 1 \:(mod\: 5) \}=\{1,6\}$\\
Subgroups generate by cycles of order $5$ in $A_5$ is Sylow $5$-subgroups.\\
$\langle (12345) \rangle, \langle (12354) \rangle, \ldots $\\
Which is more then $1$ distinct Sylowgroups which results in:\\ 
\underline{
    $\#\{$Sylow $5$-subgroups$\}=6$    
}

\section{$p=7$}
$\{ a \in S: a \equiv 1 \:(mod\: 7) \}=\{1,15\}$\\

To fullfill Third Sylow Theorem and Lagranges Theorem we must have a subgroup with $7^0=1$ elements which is $\{e\}$

%$7 \notin S \bigwedge $Lagrange's Theorem$ \Rightarrow \not{\exists}$ subgroups of order $7$.

And thus:\\
\underline{
    $\#\{$Sylow $7$-subgroups$\}=1$    
}

\end{solution}

%-----------------------
\end{document}
