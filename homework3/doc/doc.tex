\documentclass[a4paper,twoside=false,abstract=false,numbers=noenddot,
titlepage=false,headings=small,parskip=half,version=last]{scrartcl}

\usepackage[utf8]{inputenc}
\usepackage[T1]{fontenc}
\usepackage[english]{babel}

\usepackage[colorlinks=true, pdfstartview=FitV,
linkcolor=black, citecolor=black, urlcolor=blue]{hyperref}
\usepackage{verbatim}
\usepackage{graphicx}
\usepackage{multirow}

\usepackage{tikz}
\usetikzlibrary{matrix}

\usepackage{amsmath}
\usepackage{amsthm}
\usepackage{amssymb}
\usepackage{amsfonts}

\theoremstyle{definition}
\newtheorem{exercise}{Exercise}

\theoremstyle{remark}
\newtheorem*{solution}{Solution}
\newtheorem*{remark}{Remark}

\newtheorem{theorem}{Theorem}[section]
\newtheorem{lemma}[theorem]{Lemma}
\newtheorem{proposition}[theorem]{Proposition}
\newtheorem{corollary}[theorem]{Corollary}

\newcommand{\NN}{\ensuremath{\mathbb{N}}}
\newcommand{\ZZ}{\ensuremath{\mathbb{Z}}}
\newcommand{\QQ}{\ensuremath{\mathbb{Q}}}
\newcommand{\RR}{\ensuremath{\mathbb{R}}}
\newcommand{\CC}{\ensuremath{\mathbb{C}}}
\newcommand{\GG}{\ensuremath{\mathcal{G}}}
\newcommand{\Fourier}{\ensuremath{\mathcal{F}}}
\newcommand{\Laplace}{\ensuremath{\mathcal{L}}}

\DeclareMathOperator{\Hom}{Hom}
\DeclareMathOperator{\End}{End}
\DeclareMathOperator{\im}{im}
\DeclareMathOperator{\id}{id}

\renewcommand{\labelenumi}{(\alph{enumi})}

\author{Jim Holmström - 890503-7571}
\title{Groups and Rings - SF2729}
\subtitle{Homework 3}

\begin{document}

\maketitle
\thispagestyle{empty}

\begin{exercise}
{\bf
Prove that $Z(G) \leq G$ and that it's a commutative group.
}
\end{exercise}
\begin{solution}
$Z(G)$ is called the center of $G$ in algebra. \\
%\hspace{1px}\linebreak %HACK to make a newline in non-paragraph mode
%\begin{table}[h!]
%    \begin{center}
%        \begin{tabular}{r l}
%            $\GG_1$ & Associativity \\
%            $\GG_2$ & Existence of unit \\
%            $\GG_3$ & Existence of inverses \\
%        \end{tabular}
%    \end{center}
%    \caption{Group axioms}
%\end{table}

Associativity is trivially inherited from $G$.\\
$e$ satisfies $ex=xe \forall x \in G$ $\Rightarrow$ $e \in Z(G)$.\\
With $x,y \in G$ we have $(xy)g=x(yg)=x(gy)=(xg)y=(gx)y=g(xy) \forall g \in G \Rightarrow xy \in Z(G)$ i.e., $Z(G)$ is closed under the group-operation. \\  
$x \in Z(G)$ then $gx=xg \forall g \in G$ multiplying 
both from left and right in the equation with, from the original group's, inverse of 
$x$ i.e, $x'$ wich gives $x'g=gx'$ which gives $x'$ the property needed to satisfy $Z(G)$ and thus $x' \in Z(G)$ \\
This shows that $Z(G)$ is a group. $\qed$ \\

From the definition of the center we have:
$Z(G) = \{ z | \forall g \in G, zg=gz \}$ and thus we have that $xy=yx \forall x,y \in Z(G)$ 
since we have $xy=yx \forall x \in Z(G) \forall y \in G \supset Z(G)$ from the definition of the center we have that it's abelian. $\qed$

\end{solution}

%-----------------------
\begin{exercise}
{\bf
Show $Z(S_3)=\{e\}$.
}
\end{exercise}
\begin{solution}
$S_3=\{Id,(1 2),(1 3),(2 3),(1 2 3),(1 3 2)\}$. \\
$|S_3|=6=3!$\\

Knowing $\forall$ groups $G$, $Z(G)$ is commutative (and a group) and thus are $Z(S_3)$ also.
Trivial properties used thru out the calculations: $(a b)=(b a)$,$(a b)^2=Id$ and $\prod_{i=z}^b (a i)=(a b .. z)$ \\

$(a b)(a c)=(a c b)$ but $(a c)(a b)=(a b c)$ 
and thus all the elements on this form has a corresponding element 
which makes it fail to be in $Z(G)$ which is $(1 2),(2 3)$ and $(2 3)$.

Showing backwards that:
\begin{align*}
    (a b)(a b c) &\neq (a b c)(a b) \\
    (a b)(a c)(a b) &\neq (a c)(a b)(a b) = (a c) \\
    (a b)(a c) &\neq (a c)(a b)
\end{align*}
Which we have shown in previous calculations. So all elements that is on the form above is non-commutative.
$(1 2 3)$ against $(1 2)$ and $(1 3 2)$ against $(1 3)$ is non-commutative and thus $(1 2 3),(1 3 2) \notin Z(G)$ since it $\exists g \in G$ such that the commutative property doesn't hold.

This leaves us with $e$ which trivially holds the property for the center and also we know that $Z(G)$ is a group and thus must have a unit.
$Z(S_3)=\{e\} \qed$



\end{solution}


%-----------------------
\end{document}
