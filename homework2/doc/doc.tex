\documentclass[a4paper,twoside=false,abstract=false,numbers=noenddot,
titlepage=false,headings=small,parskip=half,version=last]{scrartcl}

\usepackage[utf8]{inputenc}
\usepackage[T1]{fontenc}
\usepackage[english]{babel}

\usepackage[colorlinks=true, pdfstartview=FitV,
linkcolor=black, citecolor=black, urlcolor=blue]{hyperref}
\usepackage{verbatim}
\usepackage{graphicx}
\usepackage{multirow}

\usepackage{tikz}
\usetikzlibrary{matrix}

\usepackage{amsmath}
\usepackage{amsthm}
\usepackage{amssymb}
\usepackage{amsfonts}

\theoremstyle{definition}
\newtheorem{exercise}{Exercise}

\theoremstyle{remark}
\newtheorem*{solution}{Solution}
\newtheorem*{remark}{Remark}

\newtheorem{theorem}{Theorem}[section]
\newtheorem{lemma}[theorem]{Lemma}
\newtheorem{proposition}[theorem]{Proposition}
\newtheorem{corollary}[theorem]{Corollary}

\newcommand{\NN}{\ensuremath{\mathbb{N}}}
\newcommand{\ZZ}{\ensuremath{\mathbb{Z}}}
\newcommand{\QQ}{\ensuremath{\mathbb{Q}}}
\newcommand{\RR}{\ensuremath{\mathbb{R}}}
\newcommand{\CC}{\ensuremath{\mathbb{C}}}
\newcommand{\GG}{\ensuremath{\mathcal{G}}}
\newcommand{\Fourier}{\ensuremath{\mathcal{F}}}
\newcommand{\Laplace}{\ensuremath{\mathcal{L}}}

\DeclareMathOperator{\Hom}{Hom}
\DeclareMathOperator{\End}{End}
\DeclareMathOperator{\im}{im}
\DeclareMathOperator{\id}{id}

\renewcommand{\labelenumi}{(\alph{enumi})}

\author{Jim Holmström - 890503-7571}
\title{Groups and Rings - SF2729}
\subtitle{Homework 2}

\begin{document}

\maketitle
\thispagestyle{empty}

\begin{exercise}
{\bf
Let $G$ be a finite group and let $H \subset G$ be a subset. Assume that for $\forall a,b \in H \Rightarrow ab \in H$. \\
Prove that $H$ is a subgroup.
}
\end{exercise}
\begin{solution}
\hspace{1px}\linebreak %HACK to make a newline in non-paragraph mode
\begin{table}[h!]
    \begin{center}
        \begin{tabular}{r l}
            $\GG_1$ & Associativity \\
            $\GG_2$ & Existence of unit \\
            $\GG_3$ & Existence of inverses \\
        \end{tabular}
    \end{center}
    \caption{Group axioms}
\end{table}

Associativity is trivially inherited from $G$ and therefore $H$ satisfies $\GG_1$.\\
Let $n = |H|$ and $a \in H$ then $a,a^2,\dots,a^{n+1} \in H$ since we know that $H$ is closed under the operation.
All these cannot be the same which means we have $\exists i<j : a^i=a^j$.
\begin{align*}
    a^i &= a^j \in H \\
    a^ia^{-i} &= a^ja^{-i}  \in H \\
    e &= a^{j-i}  \in H
\end{align*}
Thus $e \in H$ showing that $H$ satisfies $\GG_2$.\\
$a^{-1}=ea^{-1}=a^{j-i}a^{-1}=a^{j-i-1}$ which is $\in H$ and thus $a^{-1} \in H$ which gives that $H$ satisfies $\GG_3$.

$\therefore H \subset G$ and $H$ satisfying $\GG_{1:3} \Rightarrow H < G \qed$ 

\end{solution}

%-----------------------

\begin{exercise}
{\bf
Show that a group with no proper non-trivial subgroup is cyclic.\\
Furthermore find the order of such a group.
}
\end{exercise}
\begin{solution}
$G$ is a group with no proper non-trivial subgroup. In the case $G=\{e\}$ it's trivially cyclic. 
For the other cases we can take $a \in G,a \neq e$, we know that $<a>$ is a nontrivial subgroup of $G$ 
and since $G$ doesn't have any non-trivial proper subgroups we have $<a>=G$ and are thus cyclic.
A cyclic group is isomorphic $Z_n$ or $Z$, where for finite $G$, $n=|G|$. Since $a$ is arbitrary the 
above statement most hold $\forall a \neq e \in G$. If gcd$(a,n) \neq 1$ 
then $<a> \neq G$ and since this cannot be we have that $\forall a \in \{2,...,n-1\}$ gcd$(a,n)=1$. Therefore $n$ most be prime since that's the only number that has this property. $\qed$

\end{solution}

%-----------------------

\begin{exercise}
{\bf
Let $G$ be a group and supposed $a \in G$ generates a cyclic group of order 2 and is the unique such  element. \\
Show that $ax=xa \forall x \in G$. (Hint: consider $(xax^{-1})^2$).
}
\end{exercise}
\begin{solution}

\begin{align*}
    a &\neq e & \mbox{(Since $a$ is of order 2)} \\
    xa &\neq x & \\
    xax^{-1} &\neq e
\end{align*}
$\forall x, (xax^{-1})^2 = xaeax^{-1} = xa^2x^{-1} = \{\mbox{a of order 2 $\Rightarrow a^2 = e$ }\} = xex^{-1} = e$ \\
Thus $xax^{-1}$ is of order 2 and since $a$ is given to be the unique element of order 2 in $G$ 
we have $\forall x, xax^{-1}=a$ resulting in $\forall x, xa=ax$ $\qed$
\end{solution}

%-----------------------
\begin{exercise}
{\bf
Let $p$ and $q$ be distinct prime numbers. Find the number of generators of the cyclic group $\ZZ_{pq}$ \\
(Assuming that we have 2 typos and are therefore following the text in the book.)
}
\end{exercise}
\begin{solution}
Assuming $\ZZ_{pq} = <\ZZ_{pq},+>$ \\
Generators $ = \{ a \in \ZZ^+ : a<pq \wedge sgd(a,pq)=1 \}$
This fits the definition of eulers totient function $\phi$.\\
$|$Generators$| = \phi(pq) = (p-1)(q-1)$
$\qed$
\end{solution}

%-----------------------
\end{document}
