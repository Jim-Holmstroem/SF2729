\documentclass[a4paper,twoside=false,abstract=false,numbers=noenddot,
titlepage=false,headings=small,parskip=half,version=last]{scrartcl}

\usepackage[utf8]{inputenc}
\usepackage[T1]{fontenc}
\usepackage[english]{babel}

\usepackage[colorlinks=true, pdfstartview=FitV,
linkcolor=black, citecolor=black, urlcolor=blue]{hyperref}
\usepackage{verbatim}
\usepackage{graphicx}
\usepackage{multirow}

\usepackage{tikz}
\usetikzlibrary{matrix}

\usepackage{amsmath}
\usepackage{amsthm}
\usepackage{amssymb}
\usepackage{amsfonts}

\theoremstyle{definition}
\newtheorem{exercise}{Exercise}

\theoremstyle{remark}
\newtheorem*{solution}{Solution}
\newtheorem*{remark}{Remark}

\newtheorem{theorem}{Theorem}[section]
\newtheorem{lemma}[theorem]{Lemma}
\newtheorem{proposition}[theorem]{Proposition}
\newtheorem{corollary}[theorem]{Corollary}

\newcommand{\NN}{\ensuremath{\mathbb{N}}}
\newcommand{\ZZ}{\ensuremath{\mathbb{Z}}}
\newcommand{\QQ}{\ensuremath{\mathbb{Q}}}
\newcommand{\RR}{\ensuremath{\mathbb{R}}}
\newcommand{\CC}{\ensuremath{\mathbb{C}}}
\newcommand{\GG}{\ensuremath{\mathcal{G}}}
\newcommand{\Fourier}{\ensuremath{\mathcal{F}}}
\newcommand{\Laplace}{\ensuremath{\mathcal{L}}}

\DeclareMathOperator{\Hom}{Hom}
\DeclareMathOperator{\End}{End}
\DeclareMathOperator{\im}{im}
\DeclareMathOperator{\id}{id}

\renewcommand{\labelenumi}{(\alph{enumi})}

\author{Jim Holmström - 890503-7571}
\title{Groups and Rings - SF2729}
\subtitle{Homework 6}

\begin{document}

\maketitle
\thispagestyle{empty}

\begin{exercise}
{\bf
Let $H=\langle (12) \rangle \le S_3$ and $K=\langle (123) \rangle \le S_3$\\
Consider the $S_3$-set given by $S_3/H \times S_3/K$ Write this $S_3$-set as a
disjoint union of transitive $S_3$-sets.
}
\end{exercise}
\begin{solution}

$h=(12)$ and $k=(123)$ \\

General rules used:\\
$b\langle b \rangle=\langle b \rangle$\\
$b^{-1}\langle b \rangle=\langle b \rangle$ ($\langle b \rangle$ is cyclic)\\


\begin{equation*}
    S_3/H=\{H,(13)H,(23)H\}
\end{equation*}

\begin{equation*}
    S_3/K=\{K,(12)K\}
\end{equation*}

\begin{align*}
    S_3/H \times S_3/K  &=        & &\\
                        &\{  (H,K)&,& (    H,(12)K), \\
                        &((13)H,K)&,& ((13)H,(12)K), \\
                        &((23)H,K)&,& ((23)H,(12)K)\}
\end{align*}

Transitive if $\forall x_1,x_2 \in S_3-$set$\exists g \in S_3 : gx_1 = x_2$\\

%$h(aH,K)=(ahH=ah\langle h\rangle ,hK)=(aH,hK)$ for $a=\{e,(13),(23)\}$\\
$(12)(H,K)=(hH=h\langle h\rangle ,hK)=(H,(12)K)$\\
$(13)((13)H,(12)K)=(H,(13)(12)K=(123)K=K)=(H,K)$\\
$(23)((23)H,(12)K)=(H,(23)(12)K=(132)K=k^{-1}K=K)=(H,K)$\\
$(12)((23)H,(12)K)=((12)(23)H=(123)H=(123)(12)H=(13)H,K)=((13)H,K)\\$
$(23)((23)H,K)=(H,(23)k^{-1}K=(23)(132)K=(12)K)=(H,(12)K)$\\

Since G is closed under the operation one can take any combination of these
$g_i$ (or its inverse if you want to go in the other direction) to get from
$\forall x_1$ to $\forall x_2$ with some $g=g_2g_1 \in G$ and 
all elements are therefore transitive. (One can easily show that these are
connected by drawing an arrow from $\rightarrow$ to (and a backarrow with the
inverse permutation) the elements in the five equations above in the 
listing of the set above.)\\

So the resulting disjoint union of transitive $S_3$-sets being simply (without
union):\\
\begin{align*}
    S_3/H \times S_3/K  &=        & &\\
                        &\{  (H,K)&,& (    H,(12)K), \\
                        &((13)H,K)&,& ((13)H,(12)K), \\
                        &((23)H,K)&,& ((23)H,(12)K)\}
\end{align*}

\end{solution}

%-----------------------
\end{document}
