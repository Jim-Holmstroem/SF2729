\documentclass[a4paper,twoside=false,abstract=false,numbers=noenddot,
titlepage=false,headings=small,parskip=half,version=last]{scrartcl}

\usepackage[utf8]{inputenc}
\usepackage[T1]{fontenc}
\usepackage[english]{babel}

\usepackage[colorlinks=true, pdfstartview=FitV,
linkcolor=black, citecolor=black, urlcolor=blue]{hyperref}
\usepackage{verbatim}
\usepackage{graphicx}
\usepackage{multirow}

\usepackage{tikz}
\usetikzlibrary{matrix}

\usepackage{amsmath}
\usepackage{amsthm}
\usepackage{amssymb}
\usepackage{amsfonts}

\usepackage{enumerate}

\theoremstyle{definition}
\newtheorem{exercise}{Exercise}

\theoremstyle{remark}
\newtheorem*{solution}{Solution}
\newtheorem*{remark}{Remark}

\newtheorem{theorem}{Theorem}[section]
\newtheorem{lemma}[theorem]{Lemma}
\newtheorem{proposition}[theorem]{Proposition}
\newtheorem{corollary}[theorem]{Corollary}

\newcommand{\NN}{\ensuremath{\mathbb{N}}}
\newcommand{\ZZ}{\ensuremath{\mathbb{Z}}}
\newcommand{\QQ}{\ensuremath{\mathbb{Q}}}
\newcommand{\RR}{\ensuremath{\mathbb{R}}}
\newcommand{\CC}{\ensuremath{\mathbb{C}}}
\newcommand{\GG}{\ensuremath{\mathcal{G}}}
\newcommand{\Fourier}{\ensuremath{\mathcal{F}}}
\newcommand{\Laplace}{\ensuremath{\mathcal{L}}}

\DeclareMathOperator{\Hom}{Hom}
\DeclareMathOperator{\End}{End}
\DeclareMathOperator{\im}{im}
\DeclareMathOperator{\id}{id}

\renewcommand{\labelenumi}{(\alph{enumi})}

\author{Jim Holmström - 890503-7571}
\title{Groups and Rings - SF2729}
\subtitle{Homework 4 (rings)}

\begin{document}

\maketitle
\thispagestyle{empty}

\begin{exercise}
{\bf
}
\end{exercise}
\begin{solution}

\begin{equation}
    \label{eq:circular}
    \{ACC\Leftrightarrow MC\Leftrightarrow FBC\}
    \Leftrightarrow
    \{ACC\Rightarrow MC,MC\Rightarrow FBC,FBC\Rightarrow ACC\}
\end{equation} 

\begin{description}
    \item[$ACC\Rightarrow MC:$] 
        Instead show it by a rewrite
        \begin{equation}
            \{ACC\Rightarrow MC\}\Leftrightarrow \{\lnot{MC}\Rightarrow \lnot{ACC}\}
        \end{equation}
        If $\lnot{MC}$ then $\exists S$ of ideals with no ideal not properly
        contained in any other ideal of $S$ so the sequence of ideals extends
        to infinity $\Rightarrow \lnot{ACC}$
    \item[$MC\Rightarrow FBC:$]
        As above show it by a rewrite
        \begin{equation}
            \{MC\Rightarrow FBC\}\Leftrightarrow \{\lnot{FBC}\Rightarrow\lnot{MC}\}
        \end{equation}
        If $\lnot{FBC} \exists N \lhd R$ having no finite generating set. 
        Let
        \begin{equation}
            b_1 \in N, \langle b_1 \rangle=N_1,r \in R
        \end{equation}
        By the properties of ideal we have
        \begin{equation}
            rb_1 \in N \forall r \Leftarrow \langle b_1 \rangle=N_1 \subset N 
        \end{equation}
        $N_1 \subset N$ since otherwise it would contradict $FBC$ by $b_1$
        generating $N$.
        Let 
        \begin{equation}
            b_2\in N\ N_2, \langle b_1,b_2 \rangle =N_2
        \end{equation}
        then $N_2 \subset N$ for the same reason as for ''$N_1 \subset N$''.
        \begin{equation}
            b_2 \notin N_1 \Rightarrow N_1 \subset N_2 \subset N
        \end{equation}
        $N$ is not finitely generated we can continue with the above to
        infinity $\Rightarrow N_1 \subset N_2 \subset \ldots \subset N$
        That is $N_i \subset N_{i+1}$ which $\Rightarrow \lnot{MC}$
    \item[$FBC\Rightarrow ACC:$]
        Let the chain of ideals in R be
        \begin{equation}
            N_1 \subseteq N_2 \subseteq \dots, N_i \lhd R
        \end{equation}
        \begin{equation}
            \label{eq:intersect}
            \bigcup{N_i}=N\lhd R
        \end{equation}
        Let $B_N=\{b_i|i\in \left[1,n\right]\}$ be a finite basis for $N$ then
        suppose $b_i \in N_i$ and let $r$ be the maximum subscript $i$ then
        $B_n \subseteq N_r$ and we have (\ref{eq:intersect}) (all ideals
        containing the elements from $B_N$) $\Rightarrow N_r=R$ and 
        $N_r=N_{r+i} \Rightarrow ACC$

\end{description}
Using (\ref{eq:circular}) to show the wanted expression $\qed$


\end{solution}

%-----------------------
\begin{exercise}
{\bf
}
\end{exercise}
\begin{solution}

\begin{enumerate}[a)]
    \item
        Let $\gamma + \langle \alpha \rangle$ be a coset of
        $\ZZ\left[i\right]/\langle \alpha \rangle$. The division algorithm
        gives us 
        \begin{equation}
            \gamma=\alpha\sigma + \rho,\rho=0 \vee N(\rho)<N(\alpha)
        \end{equation}
        then
        \begin{equation}
            \gamma+\langle\alpha\rangle=\rho+\sigma\alpha+\langle\alpha\rangle
        \end{equation}
        we have $\sigma\alpha\in\langle\alpha\rangle$
        \begin{equation}
            \gamma+\langle\alpha\rangle=\rho+\langle\alpha\rangle
        \end{equation}
        thus all cosets contains a representative with norm < $N(\alpha)$.
        \begin{equation}
            \#\{a\in\ZZ\left[i\right]|norm < N(\alpha)\}<\inf \Rightarrow
            |\ZZ\left[i\right] / \langle\alpha\rangle|<\inf
        \end{equation}
        $\qed$
    \item
        Let 
        \begin{equation}
            \pi ~\text{irreducible in}~ \ZZ\left[i\right],\langle\mu\rangle 
            ~\text{ideal in}~ \ZZ\left[i\right]
            :\langle\pi\rangle\subseteq\langle\mu\rangle
        \end{equation}
        \begin{equation}
            \ZZ\left[i\right] \text{is PID} \Rightarrow \text{ideal is
            principal} ~~\forall \text{ideal}
        \end{equation}
        then $\pi\in\langle\mu\rangle$ or $\pi=\mu\beta$, $\pi$ irreducible
        $\Rightarrow$ \{$\mu$ is a unit\}$\vee$ \{$\beta$ is a unit\}.
        \begin{description}
            \item[($\mu$ is a unit)] $\langle\mu\rangle=\ZZ\left[i\right]$
            \item[($\beta$ is a unit)] $\mu=\pi\beta^{-1}$ so
            $\mu\in\langle\pi\rangle,\langle\mu\rangle=\langle\pi\rangle$
        \end{description}
        this shows that $\langle\pi\rangle$ is a maximal ideal of
        $\ZZ\left[i\right]$ $\Rightarrow \ZZ\left[i\right]/\langle\pi\rangle$
        is a field. $\qed$ 
    \item 
        \begin{enumerate}[i.]
            \item
                \begin{equation}
                    \ZZ\left[i\right]/\langle3\rangle
                \end{equation}
                $3,3i\in\langle 3\rangle \Rightarrow \forall$ cosets is on the
                form $a+bi~~~a,b\in\{0,1,2\}$ and thus there 
                $|\{0,1,2\}\times\{0,1,2\}|=3^2=9$ unique elements.
                And the characteristic is 
                \begin{equation}
                    \text{argmin}_{c>0}(\sum^c{1}=0) 
                \end{equation}
                and we have $\sum^3{1}=0$ and thus the characteristic is $3$
                $\qed$
            \item
                \begin{equation}
                    \ZZ\left[i\right]/\langle1+i\rangle
                \end{equation}
                We know from previous exercise that each coset contains
                representative with norm less than $N(1+i)=2$.\\
                The only nonzero elements with norm less than $2$ are $\{\pm
                1,\pm i\}$. Observing
                \begin{equation}
                    \pm i=\pm(-1+(1+i))
                \end{equation}
                Which gives us the possible nontrivial cosets $\pm 1+\langle 1+i \rangle$
                but we have that 
                \begin{equation}
                    1+\langle 1+i \rangle-(-1+\langle 1+i \rangle)
                    =2+\langle 1+i \rangle=(1+i)(1-i)+\langle 1+i \rangle
                \end{equation}
                Which gives us the cosets $\langle 1+i \rangle,1+\langle 1+i
                \rangle$
                and thus the ring has order $2$ and characteristic $2$ $\qed$
            \item
                \begin{equation}
                    \ZZ\left[i\right]/\langle1+2i\rangle
                \end{equation}
                We know from previous exercise that each coset contains
                representative with norm less than $N(1+2i)=5$.\\
                Elements are on the form $a+bi~~~(a,b)\in\{0,\pm 1\}^2$ 
                or $\{\pm 2\}\times\{0\}$ or $\{0\}\times\{\pm 2\}$
                Breaking out $(1+2i)$ out of all these possible we get that
                every coset has the representatives $\{0,\pm 1,\pm 2\}$ that is
                the ring has $5$ elements and characteristic of $5$ $\qed$
        \end{enumerate}

\end{enumerate}

\end{solution}

%-----------------------
\end{document}
