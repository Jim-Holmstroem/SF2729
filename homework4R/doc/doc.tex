\documentclass[a4paper,twoside=false,abstract=false,numbers=noenddot,
titlepage=false,headings=small,parskip=half,version=last]{scrartcl}

\usepackage[utf8]{inputenc}
\usepackage[T1]{fontenc}
\usepackage[english]{babel}

\usepackage[colorlinks=true, pdfstartview=FitV,
linkcolor=black, citecolor=black, urlcolor=blue]{hyperref}
\usepackage{verbatim}
\usepackage{graphicx}
\usepackage{multirow}

\usepackage{tikz}
\usetikzlibrary{matrix}

\usepackage{amsmath}
\usepackage{amsthm}
\usepackage{amssymb}
\usepackage{amsfonts}

\theoremstyle{definition}
\newtheorem{exercise}{Exercise}

\theoremstyle{remark}
\newtheorem*{solution}{Solution}
\newtheorem*{remark}{Remark}

\newtheorem{theorem}{Theorem}[section]
\newtheorem{lemma}[theorem]{Lemma}
\newtheorem{proposition}[theorem]{Proposition}
\newtheorem{corollary}[theorem]{Corollary}

\newcommand{\NN}{\ensuremath{\mathbb{N}}}
\newcommand{\ZZ}{\ensuremath{\mathbb{Z}}}
\newcommand{\QQ}{\ensuremath{\mathbb{Q}}}
\newcommand{\RR}{\ensuremath{\mathbb{R}}}
\newcommand{\CC}{\ensuremath{\mathbb{C}}}
\newcommand{\GG}{\ensuremath{\mathcal{G}}}
\newcommand{\Fourier}{\ensuremath{\mathcal{F}}}
\newcommand{\Laplace}{\ensuremath{\mathcal{L}}}

\DeclareMathOperator{\Hom}{Hom}
\DeclareMathOperator{\End}{End}
\DeclareMathOperator{\im}{im}
\DeclareMathOperator{\id}{id}

\renewcommand{\labelenumi}{(\alph{enumi})}

\author{Jim Holmström - 890503-7571}
\title{Groups and Rings - SF2729}
\subtitle{Homework 4 (rings)}

\begin{document}

\maketitle
\thispagestyle{empty}

\begin{exercise}
{\bf
}
\end{exercise}
\begin{solution}

\begin{equation}
    \{ACC\Leftrightarrow MC\Leftrightarrow FBC\}
    \Leftrightarrow
    \{ACC\Rightarrow MC,MC\Rightarrow FBC,FBC\Rightarrow ACC\}
\end{equation} 

\begin{description}
    \item[$ACC\Rightarrow MC:$] 
        Instead show it by a rewrite
        \begin{equation}
            \{ACC\Rightarrow MC\}\Leftrightarrow \{\lnot{MC}\Rightarrow \lnot{ACC}\}
        \end{equation}
        If $\lnot{MC}$ then $\exists S$ of ideals with no ideal not properly
        contained in any other ideal of $S$ so the sequence of ideals extends
        to infinity $\Rightarrow \lnot{ACC}$
    \item[$MC\Rightarrow FBC:$]
        As above show it by a rewrite
        \begin{equation}
            \{MC\Rightarrow FBC\}\Leftrightarrow \{\lnot{FBC}\Rightarrow\lnot{MC}\}
        \end{equation}
        If $\lnot{FBC}$ $\exists$ ideal $N$ in $R$ having no finite generating
        set. Let
        \begin{equation}
            b_1 \in N, \langle b_1 \rangle=N_1,r \in R
        \end{equation}
        By the properties of ideal we have
        \begin{equation}
            rb_1 \in N \forall r \Leftarrow \langle b_1 \rangle=N_1 \subset N 
        \end{equation}
        $N_1 \subset N$ since otherwise it would contradict $FBC$ by $b_1$
        generating $N$.
        Let 
        \begin{equation}
            b_2\in N\N_2$, \langle b_1,b_2 \rangle=N_2
        \end{equation}
        then $N_2 \subset N$ for the same reason as for ''$N_1 \subset N$''.
        \begin{equation}
            b_2 \notin N_1 \Rightarrow N_1 \subset N_2 \subset N
        \end{equation}
        $N$ is not finitely generated we can continue with the above to
        infinity $\Rightarrow N_1 \subset \N_2 \subset ... \subset N$
        That is $N_i \subset N_{i+1}$ which $\Rightarrow \lnot{MC}
    \item[$FBC\Rightarrow ACC:$]
                
\end{description}

\end{solution}

%-----------------------
\begin{exercise}
{\bf
}
\end{exercise}
\begin{solution}
test
\end{solution}

%-----------------------
\end{document}
