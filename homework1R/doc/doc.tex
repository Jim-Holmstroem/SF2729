\documentclass[a4paper,twoside=false,abstract=false,numbers=noenddot,
titlepage=false,headings=small,parskip=half,version=last]{scrartcl}

\usepackage[utf8]{inputenc}
\usepackage[T1]{fontenc}
\usepackage[english]{babel}

\usepackage[colorlinks=true, pdfstartview=FitV,
linkcolor=black, citecolor=black, urlcolor=blue]{hyperref}
\usepackage{verbatim}
\usepackage{graphicx}
\usepackage{multirow}

\usepackage{tikz}
\usetikzlibrary{matrix}

\usepackage{amsmath}
\usepackage{amsthm}
\usepackage{amssymb}
\usepackage{amsfonts}

\theoremstyle{definition}
\newtheorem{exercise}{Exercise}

\theoremstyle{remark}
\newtheorem*{solution}{Solution}
\newtheorem*{remark}{Remark}

\newtheorem{theorem}{Theorem}[section]
\newtheorem{lemma}[theorem]{Lemma}
\newtheorem{proposition}[theorem]{Proposition}
\newtheorem{corollary}[theorem]{Corollary}

\newcommand{\NN}{\ensuremath{\mathbb{N}}}
\newcommand{\ZZ}{\ensuremath{\mathbb{Z}}}
\newcommand{\QQ}{\ensuremath{\mathbb{Q}}}
\newcommand{\RR}{\ensuremath{\mathbb{R}}}
\newcommand{\CC}{\ensuremath{\mathbb{C}}}
\newcommand{\GG}{\ensuremath{\mathcal{G}}}
\newcommand{\Fourier}{\ensuremath{\mathcal{F}}}
\newcommand{\Laplace}{\ensuremath{\mathcal{L}}}

\DeclareMathOperator{\Hom}{Hom}
\DeclareMathOperator{\End}{End}
\DeclareMathOperator{\im}{im}
\DeclareMathOperator{\id}{id}

\renewcommand{\labelenumi}{(\alph{enumi})}

\author{Jim Holmström - 890503-7571}
\title{Groups and Rings - SF2729}
\subtitle{Homework 1 (Rings)}

\begin{document}

\maketitle
\thispagestyle{empty}

\begin{exercise}
{\bf
Let $R \in M_2(\ZZ_2)$. Prove that R has exactly $9$ divisors of $0$. Prove
that $R^* \cong S_3$.
}
\end{exercise}
\begin{solution}
An element $a \neq 0$ is a divisor of 0 $\LeftRightarrow \exists b,c \neq 0 : ba=ac=0$
Didn't find any nice things to use in this problem so I'm using the exhausting
search technique, to avoid lots of hand calculations I made a program for this.

<M2R2class>
<misc helpers>

To get blablabla isomorphism

<rest of the code>

<link to script/data>
\end{solution}

%-----------------------
\begin{exercise}
{\bf
$G=(\ZZ_{1026})^*$. Prove that $g^{18}=1 \forall g \in G$
}
\end{exercise}
\begin{solution}

Didn't find any easy solution so did it the hard-way, to avoid hand
calculations a made a script
<zncode>

<link to script/data>
\end{solution}

%-----------------------
\end{document}
