\documentclass[a4paper,twoside=false,abstract=false,numbers=noenddot,
titlepage=false,headings=small,parskip=half,version=last]{scrartcl}

\usepackage[utf8]{inputenc}
\usepackage[T1]{fontenc}
\usepackage[english]{babel}

\usepackage[colorlinks=true, pdfstartview=FitV,
linkcolor=black, citecolor=black, urlcolor=blue]{hyperref}
\usepackage{verbatim}
\usepackage{graphicx}
\usepackage{multirow}

\usepackage{tikz}
\usetikzlibrary{matrix}

\usepackage{amsmath}
\usepackage{amsthm}
\usepackage{amssymb}
\usepackage{amsfonts}

\theoremstyle{definition}
\newtheorem{exercise}{Exercise}

\theoremstyle{remark}
\newtheorem*{solution}{Solution}
\newtheorem*{remark}{Remark}

\newtheorem{theorem}{Theorem}[section]
\newtheorem{lemma}[theorem]{Lemma}
\newtheorem{proposition}[theorem]{Proposition}
\newtheorem{corollary}[theorem]{Corollary}

\newcommand{\NN}{\ensuremath{\mathbb{N}}}
\newcommand{\ZZ}{\ensuremath{\mathbb{Z}}}
\newcommand{\QQ}{\ensuremath{\mathbb{Q}}}
\newcommand{\RR}{\ensuremath{\mathbb{R}}}
\newcommand{\CC}{\ensuremath{\mathbb{C}}}
\newcommand{\GG}{\ensuremath{\mathcal{G}}}
\newcommand{\Fourier}{\ensuremath{\mathcal{F}}}
\newcommand{\Laplace}{\ensuremath{\mathcal{L}}}

\DeclareMathOperator{\Hom}{Hom}
\DeclareMathOperator{\End}{End}
\DeclareMathOperator{\im}{im}
\DeclareMathOperator{\id}{id}

\renewcommand{\labelenumi}{(\alph{enumi})}

\author{Jim Holmström - 890503-7571}
\title{Groups and Rings - SF2729}
\subtitle{Homework 1 (Rings)}

\begin{document}

\maketitle
\thispagestyle{empty}

\begin{exercise}
{\bf
Let $R \in M_2(\ZZ_2)$. Prove that R has exactly $9$ divisors of $0$. Prove
that $R^* \cong S_3$.
}
\end{exercise}
\begin{solution}
An element $a \neq 0$ is a divisor of 0 $\Leftrightarrow \exists b,c \neq 0 : ba=ac=0$
Didn't find any nice things to use in this problem so I'm using the exhausted
search technique, to avoid lots of hand calculations I made a program for this.\\

\begin{verbatim}
import operator
import copy
import itertools as itt
import math

#===================Helpers===========================================
def int2bits(i,n,zero_element=0,one_element=1):
    return list((zero_element,one_element)[i>>j & 1] for j in xrange(n-1,-1,-1))

def maplist(a,indices):
    return map(lambda i:operator.getitem(a,i),indices)

#==================Printers==========================================
def M2R2_printer(r):
    print r
    print 
    return r

def M2R2_list_printer(rs,pre_r=None):
    """
    Fundamental flaw: doesnt wrap well
    pre_r is if you want to have an start M2R2 seperated from the others being first
    """
    firstline=""
    secondline=""
    if pre_r:
        firstline+=(str(pre_r.bits[0])+str(pre_r.bits[1]))
        firstline+=" | "
        secondline+=(str(pre_r.bits[2])+str(pre_r.bits[3]))
        secondline+=" | "

    for r in rs:
        firstline+=(str(r.bits[0])+str(r.bits[1])+ " ")
    for r in rs:
        secondline+=(str(r.bits[2])+str(r.bits[3])+" ")

    print firstline
    print secondline

def print_iso(iso):
    for x,y in iso.iteritems():
        print str(x),"=",str(y)
    print "-------------"

#===================Algebra definitions of the group/ring ===============
class perm_n:
    """
    NOTE 1-indexed
    """
    perm=[] # NOTE non cyclic representation
    def __init__(self,n,elem=1):
        self.n=n
        if isinstance(elem,list):
            self.perm=elem
        else:
            assert(1<=elem<=math.factorial(n))
            self.perm=list(itt.permutations(range(1,n+1)))[elem-1]

    def __call__(self,i):
        assert(0<i<=self.n)
        return self.perm[i-1] 

    def __str__(self):
        return str(self.perm)

    def __mul__(self,other):
        assert(self.n==other.n)
        return perm_n(self.n,map(self,other.perm))

    def __eq__(self,other):
        assert(self.n==other.n)
        return all(map(operator.eq,self.perm,other.perm))

    def __ne__(self,other):
        return not operator.__eq__(self,b)

    def __hash__(self):
        return sum(map(lambda (a,k):a**k,zip(self.perm,range(1,len(self.perm)+1))))

class M2R2:
    def __init__(self,elem=0):
        """
        0->0 (important)
        one to one map (important)
        """
        if isinstance(elem,list):
            self.bits=elem
        else:
            self.bits=int2bits(elem,4)

    def __str__(self):
        return str(self.bits[0])+str(self.bits[1])+"\n"+str(self.bits[2])+str(self.bits[3]
); 

    def __eq__(self,b):
        return all(map(operator.eq,self.bits,b.bits))
    def __ne__(self,b):
        return not operator.__eq__(self,b)

    def __add__(self,other):
        return M2R2(map(operator.xor,self.bits,b.bits))
    
    def __mul__(self,other):
        a=map(operator.and_,maplist(self.bits,[0,0,2,2]),maplist(other.bits,[0,1,0,1]))
        b=map(operator.and_,maplist(self.bits,[1,1,3,3]),maplist(other.bits,[2,3,2,3]))
        return M2R2(map(operator.xor,a,b))

    def __hash__(self):
        return sum(map(lambda (a,k):([2,3][a])**k,zip(self.bits,range(1,len(self.bits)+1
))))

#=======Setup===========================================================
R = map(lambda r:M2R2(r),range(16)) #enumerate all elements in M_2(Z_2)
Zero=M2R2(0)
Rstar=copy.copy(R)
Rstar.remove(Zero) #R\{0}

print "Commutative?",all(map(lambda (a,b):a*b==b*a,itt.product(R,repeat=2)))

print "e="
e=filter(lambda i: all(map(lambda b:i*b==b,R)),R) #e*b=b \forall b \in R
assert len(e)==1 #generalized to ensure the uniqueness of e
e=e[0]
print e

#======Assignment======================================================
#TODO generalize and push to github
print "Divisors of zero"
# \exists b\neq 0 :ab=0
DOZ_left=filter(lambda a:any(map(lambda b:a*b==Zero,Rstar)),Rstar)
# \exists b\neq 0 :ba=0 
DOZ_right=filter(lambda a:any(map(lambda b:b*a==Zero,Rstar)),Rstar) 

#pickout the elements that are both left and right
DOZ=filter(lambda a:a in DOZ_left,DOZ_right) 
M2R2_list_printer(DOZ)
\end{verbatim}\\
Which returns the divisors:\\
\begin{verbatim}
Divisors of zero
00 00 00 01 01 10 10 11 11 
01 10 11 00 01 00 10 00 11 
\end{verbatim}
And they are $9$ $\qed$\\

To generate $U(M_2(Z_2))$ and find isomorphisms:\\
\begin{verbatim}
print "Group of units"
UM2R2=filter(lambda a:any(map(lambda b:a*b==b*a==e,R)),R) # \exists a:ab=ba=e
M2R2_list_printer(UM2R2)

N=3
Perms=map(lambda i:perm_n(N,i+1),range(math.factorial(N))) #in this case=S_3

#generate all possible isos
isos= map(lambda S:dict(zip(UM2R2,S)),itt.permutations(Perms,math.factorial(N)))

#filter out all isos that preserve the structure
valid_isos = filter(lambda iso: all( map(lambda (x,y):iso[x*y]==iso[x]*iso[y],
itt.product(UM2R2,repeat=2))),isos)

print "Valid isos"
map(print_iso,valid_isos)
\end{verbatim}\\
Which returns the group of units:\\
\begin{verbatim}
Group of units
01 01 10 10 11 11 
10 11 01 11 01 10

\end{verbatim}\\
and all $6$ possible isomorphisms (where the left side is a matrix and the right a non-cyclic notated permutations:\\
\begin{verbatim}
Valid isos
10
01 = (1, 2, 3)
11
01 = (3, 2, 1)
01
11 = (2, 3, 1)
10
11 = (2, 1, 3)
01
10 = (1, 3, 2)
11
10 = (3, 1, 2)
-------------
10
01 = (1, 2, 3)
11
01 = (2, 1, 3)
01
11 = (3, 1, 2)
10
11 = (3, 2, 1)
01
10 = (1, 3, 2)
11
10 = (2, 3, 1)
-------------
10
01 = (1, 2, 3)
11
01 = (1, 3, 2)
01
11 = (2, 3, 1)
10
11 = (3, 2, 1)
01
10 = (2, 1, 3)
11
10 = (3, 1, 2)
-------------
10
01 = (1, 2, 3)
11
01 = (3, 2, 1)
01
11 = (3, 1, 2)
10
11 = (1, 3, 2)
01
10 = (2, 1, 3)
11
10 = (2, 3, 1)
-------------
10
01 = (1, 2, 3)
11
01 = (2, 1, 3)
01
11 = (2, 3, 1)
10
11 = (1, 3, 2)
01
10 = (3, 2, 1)
11
10 = (3, 1, 2)
-------------
10
01 = (1, 2, 3)
11
01 = (1, 3, 2)
01
11 = (3, 1, 2)
10
11 = (2, 1, 3)
01
10 = (3, 2, 1)
11
10 = (2, 3, 1)
-------------
\end{verbatim}\\

Seems to be in this case as long as the elements has the same order ($\phi(x^n)=\phi(x)^n=e$) they can be transformed in any way and still be a isomorphism. $\qed$

The code used can be downloaded from:\\
\url{http://www.f.kth.se/~jimho/sf2729/m2r2_test.py}
\end{solution}

%-----------------------
\begin{exercise}
{\bf
$G=(\ZZ_{1026})^*$. Prove that $g^{18}=1 \forall g \in G$
}
\end{exercise}
\begin{solution}

Didn't find any easy solution so did it the hard-way, to avoid hand
calculations a made a script
<zncode>

<link to script/data>
\end{solution}

%-----------------------
\end{document}
