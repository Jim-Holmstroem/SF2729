\documentclass[a4paper,twoside=false,abstract=false,numbers=noenddot,
titlepage=false,headings=small,parskip=half,version=last]{scrartcl}

\usepackage[utf8]{inputenc}
\usepackage[T1]{fontenc}
\usepackage[english]{babel}

\usepackage[colorlinks=true, pdfstartview=FitV,
linkcolor=black, citecolor=black, urlcolor=blue]{hyperref}
\usepackage{verbatim}
\usepackage{graphicx}
\usepackage{multirow}

\usepackage{tikz}
\usetikzlibrary{matrix}

\usepackage{amsmath}
\usepackage{amsthm}
\usepackage{amssymb}
\usepackage{amsfonts}

\theoremstyle{definition}
\newtheorem{exercise}{Exercise}

\theoremstyle{remark}
\newtheorem*{solution}{Solution}
\newtheorem*{remark}{Remark}

\newtheorem{theorem}{Theorem}[section]
\newtheorem{lemma}[theorem]{Lemma}
\newtheorem{proposition}[theorem]{Proposition}
\newtheorem{corollary}[theorem]{Corollary}

\newcommand{\NN}{\ensuremath{\mathbb{N}}}
\newcommand{\ZZ}{\ensuremath{\mathbb{Z}}}
\newcommand{\QQ}{\ensuremath{\mathbb{Q}}}
\newcommand{\RR}{\ensuremath{\mathbb{R}}}
\newcommand{\CC}{\ensuremath{\mathbb{C}}}
\newcommand{\GG}{\ensuremath{\mathcal{G}}}
\newcommand{\Fourier}{\ensuremath{\mathcal{F}}}
\newcommand{\Laplace}{\ensuremath{\mathcal{L}}}

\DeclareMathOperator{\Hom}{Hom}
\DeclareMathOperator{\End}{End}
\DeclareMathOperator{\im}{im}
\DeclareMathOperator{\id}{id}

\renewcommand{\labelenumi}{(\alph{enumi})}

\author{Jim Holmström - 890503-7571}
\title{Groups and Rings - SF2729}
\subtitle{Homework 5}

\begin{document}

\maketitle
\thispagestyle{empty}

\begin{exercise}
{\bf
Let $n \ge 1$ and $S_n$ be the permutation group. 
Describe all group homomorphisms $f : S_n \rightarrow Z_3$.
}
\end{exercise}
\begin{solution}

For $n>2$:\\

Homomorphism: $\forall a,b \in G, \Phi(ab)=\Phi(a)\Phi(b) \in \ZZ_3$\\
also always holds for $a=b$ $\Rightarrow$ $\Phi(a^2)=\Phi(a)^2=\{$By matching inverse and square table of $\ZZ_3\}=\Phi(a)^{-1}$ 
and this gives us(by left multiplying with $\Phi(a)$ and using the homomorphism property)
\begin{equation}
\label{eq:cubicidentity}
    \Phi(a^3)=0
\end{equation}


$\Phi$ is a homomorphism then $\Phi \left[ G \right] \le G$ by the first isomorphism theorem.\\
The only subgroups of $\ZZ_3$ is the trivial and the non-proper one.
The first isomorphism theorem gives that $im(\Phi)=\ZZ_3 \Rightarrow ker(\Phi)=\{e\}$
Assume that $im(\Phi)=\ZZ_3$ and show a contradiction.\\
$ker(\Phi)=\{e\} \Rightarrow \bigg( \Phi(a)=0 \Rightarrow a=e \bigg)$\\
Choose a non-trivial $a : a^3 \neq e$ (that is any element with order $\neq$ 3 or 1)
But we have the property from (\ref{eq:cubicidentity}) which also holds for the above choosed $a$ which gives us:\\
$\exists b \in S_n \textbackslash \{e\} : \Phi(b)=0$ which is a contradication and the initial assumption that $im(\Phi)=\ZZ_3$ must thus be false leaving us with 
$im(\Phi)={0}$ or in other terms $\Phi \left[ S_n \right]=\{0\}$ that is all elements are mapped to $0$ which gives us with the trivial homomorphism $\Phi(a)=0 \qed$

And for $n \le 2$ simply show it "By hand" for $S_1$ and $S_2$\\

$\Phi(e)\Phi(a)\Phi(e*a)=\Phi(a) \Rightarrow \Phi(e)=e'$ Holds forall group
homomorphisms. Basicly saying that identity is mapped to identity.\\
$S_1=\{e\}$,$\Phi(e)=e'$ which for $Z_3$ is 0. \\
$S_2=\{e,(1 2)\}$,$\Phi(e)=0$ and $(1 2)$ has order 2 and thus $\Phi((1 2)(1
2)=e)=\Phi^2((1 2))=0$\\
And with $\{0^2=0,1^2=2,2^2=4=1\}$ we have that $\Phi^2=0 \Rightarrow \Phi=0$

Which show that the homomorphism must be $\Phi(a)=0$ for $S_1$ and $S_2$ also.
$\qed$

\end{solution}

%-----------------------
\begin{exercise}
{\bf
Let $G$ be a  finite group. Consider its center $Z(G) = \{ g \in G : ga=ag \forall a\in G\}$.\\
Show that if $G/Z(G)$ is cyclic, then $G$ is abelian.
}
\end{exercise}
\begin{solution}
$G/Z(G)$ is cyclic $\Rightarrow \exists g \in G/Z(G) : G/Z(G) = \langle g
\rangle$\\
Thus $\exists h \in G : G/Z(G) = \langle hZ(G) \rangle$\\
that is all cosets of $Z(G)$ is on the form $(hZ(G))^i=h^iZ(G)$. \\
$x,y \in G$ suppose $x \in h^mZ(G),y \in h^nZ(G)$ that is $x$ and $y$ belongs
to cosets. $\exists z_1,z_2 \in Z(G) : x=h^mz_1 and y=h^nz_2 $\\
\begin{align*}
    xy &= h^mz_1h^nz_2 = (z_1\in Z(G) \Rightarrow z_1 \mbox{commutes} \forall h
    \in G)\\
    xy &= h^mh^nz_1z_2 \\
    xy &= h^{m+n}z_1z_2 \\
\end{align*}
We have the same thing in the same way with $yx$ and thus $xy=yx$ $\forall x,y
\in G$ making $G$ abelian $\qed$.

\end{solution}


%-----------------------
\end{document}
