\documentclass[a4paper,11pt]{kth-mag}
\usepackage[T1]{fontenc}
\usepackage{textcomp}
\usepackage{lmodern}
\usepackage[latin1]{inputenc}
\usepackage[swedish,english]{babel}
\usepackage{modifications}
\usepackage{amsmath}
\usepackage{amsthm}
\usepackage{amssymb}
\usepackage{amsfonts}
\usepackage{graphicx}
\usepackage{tikz}
\usepackage{pgfplots}

\newcommand{\REL}{\ensuremath{\mathfrak{R}}}
\renewcommand{\SS}{\ensuremath{\mathcal{S}}}
\newcommand{\HH}{\ensuremath{\mathcal{H}}}
\newcommand{\GG}{\ensuremath{\mathcal{G}}}
\newcommand{\RR}{\ensuremath{\mathcal{R}}}


\title{A summary of Groups and Rings SF2729}

\subtitle{}
\foreigntitle{Grupper och Ringar SF2729}
\author{Jim Holmstr\"{o}m}
\date{August 2012}
\blurb{}
\trita{}
\begin{document}
\frontmatter
\pagestyle{empty}
\removepagenumbers
\maketitle
\selectlanguage{english}
\begin{abstract}
  This paper contains a brief summary of the course Groups and Rings which is
  given at KTH in Stockholm. The basis for this is the book: A First Course In
  Abstract Algebra by John B. Fraleigh.

\end{abstract}
\clearpage
\tableofcontents*
\mainmatter
\pagestyle{newchap}
\chapter{Definitions}

\section{Structures}
\subsection{Set}
\subsection{Binary algebraic structure}
\label{sec:binstructure}
Denoted by $\langle\SS,*\rangle$ is a set $\SS$ with a 
binary operator $*$ on $\SS$. 
\subsection{Group}
\label{sec:group}
A group is a binary algebraic structure \eqref{sec:binstructure} with the
following properties:\\
G1\\
G2
...

\subsection{Ring}
\label{sec:ring}
\subsection{Field}
\label{sec:field}

\section{Substructures}
\subsection{Subgroup}
\subsection{Subring}
\subsection{Ideal}

\section{Relations between structures}
\subsection{Homomorphism}
\label{sec:homo}
A map that preserves the algebraic structure between 
$\langle\SS,*\rangle,\langle\SS',*'\rangle$ by satisfying:
\begin{equation}
    \phi(x*y)=\phi(x)*'\phi(y) \quad \forall x,y\in\SS
\end{equation}

\subsection{Isomorphism}
\label{sec:iso}
A isomorphism is a homomorphism \eqref{sec:homo} with the additional property
of $\phi$ being a one-to-one map between $\SS$ and $\SS'$.

$\exists \phi$ isomorphism between 
$\langle\SS,*\rangle,\langle\SS',*'\rangle$
we say that those are isomorphic. Isomorphic is denoted by $\SS \cong \SS'$ 
omitting the $\langle\cdot,*\rangle$ since it
almost always clear which operator is considered.

\subsubsection{Showing that binary structures are isomorphic}
\begin{description}
    \item[Step 1] Define $\phi$, that is describe $\phi(a) \forall a\in\SS$
    \item[Step 2] Show that $\phi$ is one-to-one. \eqref{sec:onetoone}
    \item[Step 3] Show that $\phi$ is onto. \eqref{sec:onto}
    \item[Step 4] Show that $\phi$ is a homomorphism. \eqref{sec:homo}

\end{description}

\subsection{Group action}
\label{sec:groupaction}


\section{Bijective functions}
\subsection{Permutation}

\chapter{Useful theorems}

\section{LOREM IPSUM}
\subsection{Lagrange's Theorem}
\subsection{Cauchy's Theorem}


\chapter{Misc. Later put into some of the above chapters}

\subsection{Equivalence relation \REL}
\begin{center}
$\REL$ is an equivalence relation on a set $\SS$ \\
$\Leftrightarrow$
\end{center}

\begin{equation}\left\{
    \begin{array}{l l}
        x \REL x & \\
        x \REL y \Rightarrow y \REL x & \\
        x \REL y \wedge y \REL z \Rightarrow x \REL z & 
    \end{array} , \forall x,y,z \in \SS
\right.\end{equation}

The equivalence relation is often denoted with a \textasciitilde

\subsection{Binary operations $*$}
\begin{equation}
     \begin{array}{r l}
         * : \SS \times \SS &\rightarrow \SS \\
                       (x,y)&\mapsto *((x,y))=x*y
     \end{array}
\end{equation}

The binary function $*((x,y))$ is denoted by $x*y$.

\subsubsection{\HH\, closed under $*$}
$\HH\subseteq\SS$ closed under $*$:
\begin{equation}
    \label{eq:closed}
    a,b \in \HH \Rightarrow (a*b) \in \HH 
\end{equation}
Note that in the case $\HH=\SS$ will always be closed from the definition of
the operation.


\subsubsection{Commutative}
Commutativity is the operation property:
\begin{equation}
    \label{eq:comm}
    a*b=b*a \quad \forall a,b \in \SS 
\end{equation}

\subsubsection{Associative}
Associativity is the operation property:
\begin{equation}
    \label{eq:assoc}
    (a*b)*c = a*(b*c) \quad \forall a,b,c \in \SS
\end{equation}
which basically means that the order of operation is invariant and we might as
well denote both expressions in \eqref{eq:assoc} unambiguously without specifying
the operation order with $a*b*c$.

\section{these are grouped somehow}
\subsection{One-to-one}
\label{sec:onetoone}
A one-to-one map $\phi$ means that each element $x$ has a corresponding $\phi(x)$.

Show one-to-one by showing:
\begin{equation}
    \label{eq:onetoone}
    \phi(x)=\phi(y) \in\SS' \Rightarrow x=y \in\SS
\end{equation}
That is given the first statement in \eqref{eq:onetoone} deduce the next
statement.

\subsection{Onto}
\label{sec:onto}
$\phi$ is onto $\SS'$ if ...
Show onto by showing:
\begin{equation}
   \forall x'\exists x : \phi(x) = x' \quad x\in\SS,x'\in\SS'
\end{equation}
That is assume $x'$ is given show that $\exists ...$

\end{document}
