\documentclass[a4paper,11pt]{kth-mag}
\usepackage[T1]{fontenc}
\usepackage{textcomp}
\usepackage{lmodern}
\usepackage[latin1]{inputenc}
\usepackage[swedish,english]{babel}
\usepackage{modifications}
\usepackage{amsmath}
\usepackage{amsthm}
\usepackage{amssymb}
\usepackage{amsfonts}
\usepackage{graphicx}
\usepackage{tikz}
\usepackage{pgfplots}

\newcommand{\REL}{\ensuremath{\mathfrak{R}}}
\renewcommand{\SS}{\ensuremath{\mathcal{S}}}
\newcommand{\GG}{\ensuremath{\mathcal{G}}}
\newcommand{\RR}{\ensuremath{\mathcal{R}}}


\title{A summary of Groups and Rings SF2729}

\subtitle{}
\foreigntitle{Grupper och Ringar SF2729}
\author{Jim Holmstr\"{o}m}
\date{August 2012}
\blurb{}
\trita{}
\begin{document}
\frontmatter
\pagestyle{empty}
\removepagenumbers
\maketitle
\selectlanguage{english}
\begin{abstract}
  This paper contains a brief summary of the course Groups and Rings which is
  given at KTH in Stockholm. The basis for this is the book: A First Course In
  Abstract Algebra by John B. Fraleigh.

\end{abstract}
\clearpage
\tableofcontents*
\mainmatter
\pagestyle{newchap}
\chapter{Definitions}

\section{Structures}
\subsection{Set}
\subsection{Group}
\subsection{Ring}
\subsection{Field}

\section{Substructures}
\subsection{Subgroup}
\subsection{Subring}
\subsection{Ideal}

\section{Relations between structures}
\subsection{Homomorphism}
\subsection{Isomorphism}
\subsection{Group action}

\section{Bijective functions}
\subsection{Permutation}

\chapter{Useful theorems}

\section{LOREM IPSUM}
\subsection{Lagrange's Theorem}
\subsection{Cauchy's Theorem}


\chapter{Misc. Later put into some of the above chapters}

\subsection{Equivalence relation \REL}
\begin{center}
\REL~ is an equivalence relation on a set \SS~ \\
$\Leftrightarrow$
\end{center}

\begin{equation}\left\{
    \begin{array}{l l}
        x \REL x & \\
        x \REL y \Rightarrow y \REL x & \\
        x \REL y \wedge y \REL z \Rightarrow x \REL z & 
    \end{array} , \forall x,y,z \in \SS
\right.\end{equation}

\subsection{Binary operations $*$}
\begin{equation}
     \begin{array}{r l}
         * : \SS \times \SS &\rightarrow \SS \\
                       (x,y)&\mapsto *((x,y))=x*y
     \end{array}
\end{equation}

The binary function $*((x,y))$ is denoted by $x*y$.

\subsubsection{Closed under *}

\subsubsection{Commutative}

\subsubsection{Associative}
\begin{equation}
    (a*b)*c = a*(b*c)
\end{equation}
Which gives us 

\end{document}
