\documentclass[a4paper,twoside=false,abstract=false,numbers=noenddot,
titlepage=false,headings=small,parskip=half,version=last]{scrartcl}

\usepackage[utf8]{inputenc}
\usepackage[T1]{fontenc}
\usepackage[english]{babel}

\usepackage[colorlinks=true, pdfstartview=FitV,
linkcolor=black, citecolor=black, urlcolor=blue]{hyperref}
\usepackage{verbatim}
\usepackage{graphicx}
\usepackage{multirow}

\usepackage{tikz}
\usetikzlibrary{matrix}

\usepackage{amsmath}
\usepackage{amsthm}
\usepackage{amssymb}
\usepackage{amsfonts}

\theoremstyle{definition}
\newtheorem{exercise}{Exercise}

\theoremstyle{remark}
\newtheorem*{solution}{Solution}
\newtheorem*{remark}{Remark}

\newtheorem{theorem}{Theorem}[section]
\newtheorem{lemma}[theorem]{Lemma}
\newtheorem{proposition}[theorem]{Proposition}
\newtheorem{corollary}[theorem]{Corollary}

\newcommand{\NN}{\ensuremath{\mathbb{N}}}
\newcommand{\ZZ}{\ensuremath{\mathbb{Z}}}
\newcommand{\QQ}{\ensuremath{\mathbb{Q}}}
\newcommand{\RR}{\ensuremath{\mathbb{R}}}
\newcommand{\CC}{\ensuremath{\mathbb{C}}}
\newcommand{\GG}{\ensuremath{\mathcal{G}}}
\newcommand{\Fourier}{\ensuremath{\mathcal{F}}}
\newcommand{\Laplace}{\ensuremath{\mathcal{L}}}

\DeclareMathOperator{\Hom}{Hom}
\DeclareMathOperator{\End}{End}
\DeclareMathOperator{\im}{im}
\DeclareMathOperator{\id}{id}

\renewcommand{\labelenumi}{(\alph{enumi})}

\newcommand{\poly}[4]{#1_{0#4}+#1_{1#4}{#2}+#1_{2#4}{#2}^2+\dots+#1_{#3#4}{#2}^{#3}}
\newcommand{\polymod}[4]{#1_{{#4}0}+#1_{{#4}1}{#2}+#1_{{#4}2}{#2}^2+\dots+#1_{#4#3}{#2}^{#3}}

\author{Jim Holmström - 890503-7571}
\title{Groups and Rings - SF2729}
\subtitle{Homework 5 (Rings)}

\begin{document}

\maketitle
\thispagestyle{empty}

\begin{exercise}
{\bf
    Let $E$ be an extension field of $F$, and let $\alpha,\beta \in E$. Suppose
    $\alpha$ is transcendental over $F$ but algebraic over $F(\beta)$. Show that
    $\beta$ is algebraic $F(\alpha)$.
}
\end{exercise}
\begin{solution}
    $\alpha$ algebraic over $F(\beta) \Rightarrow \exists \alpha :
    poly(\alpha)=0,poly\neq0$, generally
    \begin{equation}
        \poly{c}{\alpha}{n}{}=0,c_i \in F(\beta)
    \end{equation}
    \begin{equation}
        c_i \in F(\beta) \Leftrightarrow c_i=\poly{c}{\beta}{m}{i},c_{ij}\in F
    \end{equation}
    By observing the expressions in index notation
    \begin{equation}
        \sum_j{\left(\sum_i{c_{ij}\beta^i}\right)\alpha^j}=
        \sum_{i,j}{c_{ij}\beta^i\alpha^j}=
        \sum_{i,j}{c_{ij}\alpha^j\beta^i}=
        \sum_i{\left(\sum_j{c_{ij}\alpha^j}\right)\beta^j}
    \end{equation}
    and written out more explicity it becomes
    \begin{equation}
        \begin{split}
            & (\poly{c}{\beta}{m}{0}) + \\
            & (\poly{c}{\beta}{m}{1})\alpha + \\
            & (\poly{c}{\beta}{m}{2})\alpha^2 + \\
            &\dots \\
            & (\poly{c}{\beta}{m}{n})\alpha^n
        \end{split}
    \end{equation}
    rearranged to the expression
    \begin{equation}
        \begin{split}
            & (\polymod{c}{\alpha}{n}{0}) + \\
            & (\polymod{c}{\alpha}{n}{1})\beta + \\
            & (\polymod{c}{\alpha}{n}{2})\beta^2 + \\
            &\dots \\
            & (\polymod{c}{\alpha}{n}{m})\beta^m
        \end{split}
    \end{equation}
    going back to the index notation we have
    \begin{equation}
        \left(\sum_j{c_{ij}\alpha^j}\right) \in F(\alpha) 
    \end{equation}
    and this shows that $\beta$ is algebraic over $F(\alpha) \qed$
    \end{solution}
%-----------------------
\begin{exercise}
{\bf
    Let $E$ be a finite extension field of $F$. Let $D$ be an integral domain :
    $F\subseteq D\subseteq E$. Show that $D$ is a field.
}
\end{exercise}
\begin{solution}
    To show this we only need to show that $\alpha \in D\backslash \{0\}
    \Rightarrow \alpha^{-1} \in D$ since we know that $D$ is commutative 
    and has unity from it's integral domain properties. 
    $E$ is finite extension over $F \Rightarrow \alpha$ algebraic over $F$.
    With $deg(\alpha,F)=n$ Theorem 30.23 gives us
    \begin{equation}
        F(\alpha) = \{ a_0+a_1\alpha+a_2\alpha^2+\dots+a_{n-1}\alpha^{n-1} |a_i\in F\}
    \end{equation}
    $\alpha^{-1}\in F(\alpha)$, that is $\alpha^{-1}$ can be written as a
    polynomial of $\alpha$ with coeffs in $F$, and is in $D$ $\qed$
\end{solution}

%-----------------------
\end{document}
