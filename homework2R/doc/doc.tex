\documentclass[a4paper,twoside=false,abstract=false,numbers=noenddot,
titlepage=false,headings=small,parskip=half,version=last]{scrartcl}

\usepackage[utf8]{inputenc}
\usepackage[T1]{fontenc}
\usepackage[english]{babel}

\usepackage[colorlinks=true, pdfstartview=FitV,
linkcolor=black, citecolor=black, urlcolor=blue]{hyperref}
\usepackage{verbatim}
\usepackage{graphicx}
\usepackage{multirow}

\usepackage{tikz}
\usetikzlibrary{matrix}

\usepackage{amsmath}
\usepackage{amsthm}
\usepackage{amssymb}
\usepackage{amsfonts}

\theoremstyle{definition}
\newtheorem{exercise}{Exercise}

\theoremstyle{remark}
\newtheorem*{solution}{Solution}
\newtheorem*{remark}{Remark}

\newtheorem{theorem}{Theorem}[section]
\newtheorem{lemma}[theorem]{Lemma}
\newtheorem{proposition}[theorem]{Proposition}
\newtheorem{corollary}[theorem]{Corollary}

\newcommand{\NN}{\ensuremath{\mathbb{N}}}
\newcommand{\ZZ}{\ensuremath{\mathbb{Z}}}
\newcommand{\QQ}{\ensuremath{\mathbb{Q}}}
\newcommand{\RR}{\ensuremath{\mathbb{R}}}
\newcommand{\CC}{\ensuremath{\mathbb{C}}}
\newcommand{\GG}{\ensuremath{\mathcal{G}}}
\newcommand{\Fourier}{\ensuremath{\mathcal{F}}}
\newcommand{\Laplace}{\ensuremath{\mathcal{L}}}

\DeclareMathOperator{\Hom}{Hom}
\DeclareMathOperator{\End}{End}
\DeclareMathOperator{\im}{im}
\DeclareMathOperator{\id}{id}

\renewcommand{\labelenumi}{(\alph{enumi})}

\newcounter{qc}

\author{Jim Holmström - 890503-7571}
\title{Groups and Rings - SF2729}
\subtitle{Homework 2 (Rings)}

\begin{document}

\maketitle
\thispagestyle{empty}

\begin{exercise}
{\bf
Let $\sigma_m : \ZZ \rightarrow \ZZ_m$ be the natural homomorphism given by
$\sigma_m(a)= a~(mod~m)$.
\begin{list}
{\alph{qc}.}
{
    \usecounter{qc}
    \bfseries
    \setlength\labelwidth{3in}
}
    \item Show that
    $\overline{\sigma_m}:\ZZ\left[x\right]\rightarrow\ZZ_m\left[x\right]$ given
    by
    \begin{equation}
        \overline{\sigma_m}(a_0+a_1x+\hdots+a_nx^n)=
        \sigma_m(a_0)+\sigma_m(a_1)x+\hdots+\sigma_m(a_n)x^n
    \end{equation}
    is an homomorphism of $\ZZ\left[x\right]$ onto $\ZZ_m\left[x\right]$.
    \item Show that
    $degree(f(x)\in\ZZ\left[x\right])=degree(\overline{\sigma_m}(f(x)))=n
    \bigwedge \overline{\sigma_m(f(x))}$ has no nontrivial factors in
    $\ZZ_m\left[x\right] \Rightarrow f(x)$ is irreducible in $\QQ\left[x\right]$.
    \item Show that $x^3+17x+36$ is irreducible in $\QQ\left[x\right]$
\end{list}
}
\end{exercise}
\begin{solution}
\begin{list}
{\alph{qc}.}
{
    \usecounter{qc}
    \setlength\labelwidth{3in}
}
    \item 
        $
        \overline{\sigma_m}(f(x)+g(x)) = 
        \overline{\sigma_m}\sum{(f_i+g_i)x^i} =
        \sum{\overline{\sigma_m}(f_i+g_i)x^i} =
        \sum{(\overline{\sigma_m}(f_i)+\overline{\sigma_m}(g_i))x^i} = 
        \overline{\sigma_m}(f(x))+\overline{\sigma_m}(g(x))
        $
        and
        $
        \overline{\sigma_m}(f(x)g(x)) =
        \overline{\sigma_m}\left(
                                \sum
                                    \left(
                                        \sum f_ig_{n-i} 
                                    \right)
                                    x^n 
                            \right) 
        =
        \sum \overline{\sigma_m}
                            \left(
                                \sum f_ig_{n-i}
                            \right)x^n
        =
        \sum \left(
                    \sum \overline{\sigma_m}(f_ig_{n-i})
            \right) x^n
        =
        \sum \left(
                    \sum \overline{\sigma_m}(f_i)\overline{\sigma_m}(g_{n-i})
            \right) x^n
        =
        \overline{\sigma_m}(f(x))\overline{\sigma_m}(g(x))
        $
        Which shows that $\overline{\sigma_m}$ is an homomorphism.\\
        $a(x)\in \ZZ_m\left[x\right]$ and $b(x)\in \ZZ\left[x\right]$ having the
        same coeffs but seen as in \ZZ instead of $\ZZ_m$ with this we see that
        $\overline{\sigma_m}(a(x))=b(x)$, so it is onto. $\qed$
    \item
        $f=gh$ for $g,h\in\ZZ\left[x\right]$ where 
        $degree(f)>degree(g) \wedge degree(f)>degree(h)$\\
        Applying $\overline{\sigma_m}$ on $f$:
        $\overline{\sigma_m}(f)=\overline{\sigma_m}(g)\overline{\sigma_m}(h)$
        is a factorization of $\overline{\sigma_m}$ into polynoms with a degree
        less then $n$ of $\overline{\sigma_m}(f)$ which is a contradiction\\
        $\Rightarrow$ $f(x)$ is irreducible in $\ZZ\left[x\right]$\\
        $\Rightarrow$ (by Theorem 23.11) $f(x)$ is irreducible in $\QQ\left[x\right]$
        $\qed$
    \item 
        Magically choosing $m=5$\\
        $\overline{\sigma_5}(x^3+17x+36)=x^3+2x+1$\\
        By hand it's simple to show that:\\
        \begin{equation}
            (x^3+2x+1)(\{-2,-1,0,1,2\}) \neq 0
        \end{equation}
        and by Theorem 23.10 irreducible over $\ZZ_5$ and by the findings in (b)
        we also have that $x^3+17x+36$ is irreducible over $\QQ$
        $\qed$

\end{list}

\end{solution}

%-----------------------
\begin{exercise}
{\bf
Let $f(X)=X^4-X^2+1$. Prove that $f(X)$ is irreducible in $\ZZ\left[X\right]$
and show that $f(X)$ is reducible in $\ZZ_m\left[X\right]$ for
$m=\{2,3,5\}$ by determining the factorization into a product of
irreducible polynomials.
}
\end{exercise}
\begin{solution}
Starting with the smaller rings:\\
m=2:\\
$(x^2+x+1)^2=x^4-x^2+1$
m=3:\\
$(x^2+1)^2=x^4-x^2+1$
m=5:\\
$(x^2+3x+1)(x^2+2x+4)=x^4-x^2+1$
Which are all found by a computer program (see last in document) and hand verified to so that the
calculations is correct.\\

Now prove that $f(X)$ is irreducible in $\ZZ\left[X\right]$. Firstly noting
that I don't have infinite RAM nor infinite time so are abandoning the
computer program for this part.\\
For $f$ to have a
zero in \ZZ ~ it must divide $1$, so the only $2$ possibilities are the units and we
have that $f(1)=f(-1)=1\neq0$. Which results in no factors of degree 1.\\
Now look for factors of degree $2$:\\
Assume factors and exists and expand the polynoms with general coeffs in \ZZ:\\
\begin{equation}
    (a_2x^2+a_1x+a_0)(b_2x^2+b_1x+b_0)=x^4-x^2+1
\end{equation}
Calculate the left side (by Cauchy-product) and pattern-match the coeffs:\\
\begin{eqnarray}
    a_0b_0&=&1 \\
    a_0b_1+a_1b_0&=&0\\
    a_0b_2+a_1b_1+a_2b_0&=&-1\\
    a_2b_1+a_1b_2&=&0\\ 
    a_2b_2&=&1
\end{eqnarray}
One can see that this system is not solvable in \ZZ~ since from the first $2$
equations we have $(a_0=b_0=1 \bigvee a_0=b_0=-1)\bigwedge(a_2=b_2=1\bigvee
a_2=b_2=-1)$\\ and testing all these possible combinations of $a_0,b_0,a_2,b_2$
in the three last equations will all be insolvable in \ZZ~ and thus leaving us
with that $f$ can't be factored in \ZZ~ by definition irreducible in
$\ZZ\left[X\right] \qed$

\end{solution}
\newpage
The computer-program in use:\\
\begin{verbatim}
import operator
import copy
import itertools as itt
import string
import math

#=======Ring definition===================
class Zn:
    def __init__(self,n,i):
        """
        Initz Z_n with the element i
        """
        self.n=n
        self.i=i%n  #fugly but works with negative numbers which is nice i
                    #(but platform dependent perhaps)

    def __str__(self):
        """
        You are on your own on tracking n, mostly one has the same n
        """
        return str(self.i)

    def __eq__(self,other):
        """
        NOOOOT!! Must be of the same Zn to be the same
        """
        if(isinstance(other,int)):
            return self.i==other
        return self.i==other.i
    def __ne__(self,other):
        return not operator.__eq__(self,other)
    
    def __add__(self,other):
        assert self.n==other.n
        return Zn(self.n,(self.i+other.i)%self.n)
    def __mul__(self,other):
        assert self.n==other.n #not defined else
        return Zn(self.n,(self.i*other.i)%self.n)
    def __pow__(self,m):
        """
        return g**n
        """
        return Zn(self.n,(self.i**m)%self.n)
    
    def __hash__(self):
        return self.i

class Polynom:
    def __init__(self,c):
        """
        Starts with the constant c_0
        """
        self.c=c

    def __str__(self):
        output=""
        for i,c_i in enumerate(reversed(self.c)):
            if((len(self)-i-1)>1): #fugly with double reverse TODO fix
                output+=str(c_i)
                output+="X^"+str(len(self)-i-1)+"+"
            elif((len(self)-i-1)==1):
                output+=str(c_i)+"X+"
            else:
                output+=str(c_i)
        return output

    def __eq__(self,other):
        N=len(self)
        M=len(other)
        #if all elements is equal and the part hanging outside is all zero
        #then the polynoms are equal
        pseudoeq=all(map(lambda (a,b):a==b,zip(self.c,other.c)))
        if(N==M):#fugly code #TODO fixit 
            return pseudoeq 
        elif(M<N):
            return pseudoeq and reduce(operator.add,self.c[M:N])==0
        else:#(N<M)
            return pseudoeq and reduce(operator.add,other.c[N:M])==0

    def __neq__(self,other):
        return not operator.__eq__(self,other)

    def __add__(self,other):
        c_res=map(lambda (i,j):i+j,itt.izip_longest(self.c,other.c,fillvalue=0))
        return Polynom(c_res)
    
    def __len__(self):
        return len(self.c)

    def __mul__(self,other):
        #cauchy
        c_n=[0]*(len(self)+len(other)-1)

        for k in range(len(c_n)): 
            filtered=filter(lambda (i,j):i+j==k,itt.product(range(len(self)),
            range(len(other))))
            c_n[k] = reduce(operator.add,map(lambda (i,j): 
            self.c[i]*other.c[j],filtered))
        
        return Polynom(c_n)

degree=3
for m in [2,3,5]:
    print "m=",m,",degree",degree,":"
    Zm=map(lambda i:Zn(m,i),range(m))

    PZm=map(lambda c:Polynom(c),itt.product(Zm,repeat=(degree+1)))

    F=Polynom(map(lambda i:Zn(m,i),[1,0,-1,0,1])) 

    factors=filter(lambda (f,g):f*g==F,itt.product(PZm,repeat=2))

    for f,g in factors:
        print str(f)+"*"+str(g)+"="+str(f*g)

\end{verbatim}

%-----------------------
\end{document}
