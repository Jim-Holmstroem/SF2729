\documentclass[a4paper,twoside=false,abstract=false,numbers=noenddot,
titlepage=false,headings=small,parskip=half,version=last]{scrartcl}

\usepackage[utf8]{inputenc}
\usepackage[T1]{fontenc}
\usepackage[english]{babel}

\usepackage[colorlinks=true, pdfstartview=FitV,
linkcolor=black, citecolor=black, urlcolor=blue]{hyperref}
\usepackage{verbatim}
\usepackage{graphicx}
\usepackage{multirow}

\usepackage{tikz}
\usetikzlibrary{matrix}

\usepackage{amsmath}
\usepackage{amsthm}
\usepackage{amssymb}
\usepackage{amsfonts}

\theoremstyle{definition}
\newtheorem{exercise}{Exercise}

\theoremstyle{remark}
\newtheorem*{solution}{Solution}
\newtheorem*{remark}{Remark}

\newtheorem{theorem}{Theorem}[section]
\newtheorem{lemma}[theorem]{Lemma}
\newtheorem{proposition}[theorem]{Proposition}
\newtheorem{corollary}[theorem]{Corollary}

\newcommand{\NN}{\ensuremath{\mathbb{N}}}
\newcommand{\ZZ}{\ensuremath{\mathbb{Z}}}
\newcommand{\QQ}{\ensuremath{\mathbb{Q}}}
\newcommand{\RR}{\ensuremath{\mathbb{R}}}
\newcommand{\CC}{\ensuremath{\mathbb{C}}}
\newcommand{\GG}{\ensuremath{\mathcal{G}}}
\newcommand{\Fourier}{\ensuremath{\mathcal{F}}}
\newcommand{\Laplace}{\ensuremath{\mathcal{L}}}

\DeclareMathOperator{\Hom}{Hom}
\DeclareMathOperator{\End}{End}
\DeclareMathOperator{\im}{im}
\DeclareMathOperator{\id}{id}

\renewcommand{\labelenumi}{(\alph{enumi})}

\newcounter{qc}

\author{Jim Holmström - 890503-7571}
\title{Groups and Rings - SF2729}
\subtitle{Homework 2 (Rings)}

\begin{document}

\maketitle
\thispagestyle{empty}

\begin{exercise}
{\bf
Let $\sigma_m : \ZZ \rightarrow \ZZ_m$ be the natural homomorphism given by
$\sigma_m(a)= a~(mod~m)$.
\begin{list}
{\alph{qc}.}
{
    \usecounter{qc}
    \bfseries
    \setlength\labelwidth{3in}
}
    \item Show that
    $\overline{\sigma_m}:\ZZ\left[x\right]\rightarrow\ZZ_m\left[x\right]$ given
    by
    \begin{equation}
        \overline{\sigma_m}(a_0+a_1x+\hdots+a_nx^n)=
        \sigma_m(a_0)+\sigma_m(a_1)x+\hdots+\sigma_m(a_n)x^n
    \end{equation}
    is an homomorphism of $\ZZ\left[x\right]$ onto $\ZZ_m\left[x\right]$.
    \item Show that
    $degree(f(x)\in\ZZ\left[x\right])=degree(\overline{\sigma_m}(f(x)))=n
    \bigwedge \overline{\sigma_m(f(x))}$ has no nontrivial factors in
    $\ZZ_m\left[x\right] \Rightarrow f(x)$ is irreducible in $\QQ\left[x\right]$.
    \item Show that $x^3+17x+36$ is irreducible in $\QQ\left[x\right]$
\end{list}
}
\end{exercise}
\begin{solution}
\begin{list}
{\alph{qc}.}
{
    \usecounter{qc}
    \setlength\labelwidth{3in}
}
    \item 
        $
        \overline{\sigma_m}(f(x)+g(x)) = 
        \overline{\sigma_m}\sum{(f_i+g_i)x^i} =
        \sum{\overline{\sigma_m}(f_i+g_i)x^i} =
        \sum{(\overline{\sigma_m}(f_i)+\overline{\sigma_m}(g_i))x^i} = 
        \overline{\sigma_m}(f(x))+\overline{\sigma_m}(g(x))
        $
        and
        $
        \overline{\sigma_m}(f(x)g(x)) =
        \overline{\sigma_m}\left(
                                \sum
                                    \left(
                                        \sum f_ig_{n-i} 
                                    \right)
                                    x^n 
                            \right) 
        =
        \sum \overline{\sigma_m}
                            \left(
                                \sum f_ig_{n-i}
                            \right)x^n
        =
        \sum \left(
                    \sum \overline{\sigma_m}(f_ig_{n-i})
            \right) x^n
        =
        \sum \left(
                    \sum \overline{\sigma_m}(f_i)\overline{\sigma_m}(g_{n-i})
            \right) x^n
        =
        \overline{\sigma_m}(f(x))\overline{\sigma_m}(g(x))
        $
        Which shows that $\overline{\sigma_m}$ is an homomorphism.\\
        $a(x)\in \ZZ_m\left[x\right]$ and $b(x)\in \ZZ\left[x\right]$ having the
        same coeffs but seen as in \ZZ instead of $\ZZ_m$ with this we see that
        $\overline{\sigma_m}(a(x))=b(x)$, so it is onto. $\qed$
    \item
        $f=gh$ for $g,h\in\ZZ\left[x\right]$ where 
        $degree(f)>degree(g) \wedge degree(f)>degree(h)$\\
        Applying $\overline{\sigma_m}$ on $f$:
        $\overline{\sigma_m}(f)=\overline{\sigma_m}(g)\overline{\sigma_m}(h)$
        is a factorization of $\overline{\sigma_m}$ into polynoms with a degree
        less then $n$ of $\overline{\sigma_m}(f)$ which is a contradiction\\
        $\Rightarrow$ $f(x)$ is irreducible in $\ZZ\left[x\right]$\\
        $\Rightarrow$ (by Theorem 23.11) $f(x)$ is irreducible in $\QQ\left[x\right]$
        $\qed$
    \item 
        Magically choosing $m=5$\\
        $\overline{\sigma_5}(x^3+17x+36)=x^3+2x+1$\\
        By hand it's simple to show that:\\
        \begin{equation}
            (x^3+2x+1)(\{-2,-1,0,1,2\}) \neq 0
        \end{equation}
        and by Theorem 23.10 irreducible over $\ZZ_5$ and by the findings in (b)
        we also have that $x^3+17x+36$ is irreducible over $\QQ$
        $\qed$

\end{list}

\end{solution}

%-----------------------
\begin{exercise}
{\bf
Let $f(X)=X^4-X^2+1$. Prove that $f(X)$ is irreducible in $\ZZ\left[X\right]$
and show that $f(X)$ is reducible in $\ZZ_m\left[X\right]$ for
$m=\{2,3,5\}$ by determining the factorization into a product of
irreducible polynomials.
}
\end{exercise}
\begin{solution}

\end{solution}

%-----------------------
\end{document}
